%%%%%%%%%%%%%%%%%%%%%%%%%%%%%%%%%%%%%%%%%
% Medium Length Professional CV
% LaTeX Template
% Version 2.0 (8/5/13)
%
% This template has been downloaded from:
% http://www.LaTeXTemplates.com
%
% Original author:
% Trey Hunner (http://www.treyhunner.com/)
%
% Important note:
% This template requires the resume.cls file to be in the same directory as the
% .tex file. The resume.cls file provides the resume style used for structuring the
% document.
%
%%%%%%%%%%%%%%%%%%%%%%%%%%%%%%%%%%%%%%%%%

%----------------------------------------------------------------------------------------
%	PACKAGES AND OTHER DOCUMENT CONFIGURATIONS
%----------------------------------------------------------------------------------------

\documentclass{resume} % Use the custom resume.cls style

\usepackage[left=0.75in,top=0.6in,right=0.75in,bottom=0.6in]{geometry} % Document margins
\usepackage{hyperref} 

\name{Pavel Yakovlev} % Your name
\address{(707)~$\cdot$~971~$\cdot$~9336 \\ yvpavel@gmail.com} % Your phone number and email
\address{\href{https://www.paulyakovlev.com}{paulyakovlev.com} \\ \href{https://www.github.com/paulyakovlev}{github.com/paulyakovlev}} % Your phone number and email


\begin{document}

%----------------------------------------------------------------------------------------
%	EDUCATION SECTION
%----------------------------------------------------------------------------------------

\begin{rSection}{Education}

    {\bf University of California, Santa Cruz} \hfill {\em September 2018 - June 2020} \\ 
    Bachelor's in Computer Science

    {\bf Santa Rosa Junior College} \hfill {\em September 2015 - May 2018}

\end{rSection}

%----------------------------------------------------------------------------------------
%	WORK EXPERIENCE SECTION
%----------------------------------------------------------------------------------------

\begin{rSection}{Experience}
    \begin{rSubsection}{Long Marine Mammal Lab}{January 2019 - Present}{Software Engineer}{Santa Cruz, CA}
        \item Developed React web app used by marine lab technicians to map and analyze geospatial data with React
        \item Built backend infastructure for data storage and retrieval.
        \item Led the development of data management process and workflow.
        \item Provided guidance for software engineering best practices. (test-driven-development, Agile-Scrum process, version control)
    \end{rSubsection}

    \begin{rSubsection}{Lawrence Livermore National Lab}{June 2019 - September 2019}{Computing Intern}{Livermore, CA}
        \item Worked on the enhancement and modernization of automated testing and software integration framework for a very large software system.
        \item Unified the build processes around industry-standard continuous integration and test-driven development practices.
    \end{rSubsection}

    \begin{rSubsection}{remote.it}{May 2018 - June 2019}{Sofware Engineering Intern}{Palo Alto, CA}
        \item Built an administrative web dashboard with React.
        \item Worked with Terraform + Packer for AWS EC2 AMI deployment and testing.
        \item Created a library of user-end Python and Shell scripts for bulk device management.
        \item Worked remotely from September 2018 through the end of the internship.
    \end{rSubsection}

\end{rSection}

%----------------------------------------------------------------------------------------
%	PROJECTS SECTION
%----------------------------------------------------------------------------------------

\begin{rSection}{Projects}

    \begin{rSubsection}{Pollen Planter (CruzHacks '20 Winner)}{\href{https://www.devpost.com/software/pollenplanter}{devpost.com/software/pollenplanter}}{}{}
        \item Mobile app that helps to support local pollinator(bees,butterflies) communities.
        \item Built with React Native + Firebase Firestore.
        \item Worked in team of 4 - took the 1st place prize for Best Earth Hack.
    \end{rSubsection}

    \begin{rSubsection}{NetGuard (Treehacks '19)}{\href{https://www.devpost.com/software/netguard}{devpost.com/software/netguard}}{}{}
        \item Alerts owner of IoT device of cyber attack and locks down the device.
        \item Written in Python, runs on Raspberry Pi, SMS alerts achieved with Twilio API.
        \item Worked with another team member to write the listener and API request component.
    \end{rSubsection}

\end{rSection}

\begin{rSection}{Key Skills}

\end{rSection}


%----------------------------------------------------------------------------------------
%	EXAMPLE SECTION
%----------------------------------------------------------------------------------------

%\begin{rSection}{Section Name}

%Section content\ldots

%\end{rSection}

%----------------------------------------------------------------------------------------

\end{document}
